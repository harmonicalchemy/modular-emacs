%%%% ~~~~~~~~~~~~~~~~~~~~~~~~~~~~~~~~~~~~~~~~~~~~~~~~~~~~~~~~~~~~~~~~~~~~~~~~~~~~
%%   File:        ~/.emacs.d/Docs/TeX/ref-book-setup.tex
%%   Ref:         <https://github.com/harmonicalchemy/modular-emacs>
%%   Purpose:     LaTeX Setup Include File for Reference Books
%%   Author:      Alisha Awen
%%   Maintainer:  Alisha Awen

%%   This program is free software: you can redistribute it and/or modify
%%   it under the terms of the GNU General Public License as published by
%%   the Free Software Foundation, either version 3 of the License, or
%%   (at your option) any later version.

%%   This program is distributed in the hope that it will be useful,
%%   but WITHOUT ANY WARRANTY; without even the implied warranty of
%%   MERCHANTABILITY or FITNESS FOR A PARTICULAR PURPOSE.  See the
%%   GNU General Public License for more details.

%%   You should have received a copy of the GNU General Public License
%%   along with this program.  If not, see <https://www.gnu.org/licenses/>.

%%   To Customize: Copy/Clone ~/.emacs.d/Docs/PubOps/org-templates (which contains
%%   this file) to your .org doc's MASTER FOLDER. Then, modify your cloned copy of
%%   this file...
%%%% ~~~~~~~~~~~~~~~~~~~~~~~~~~~~~~~~~~~~~~~~~~~~~~~~~~~~~~~~~~~~~~~~~~~~~~~~~~~~

%%% ~~~~~~~~~~~~~~~~~~~~~~~~~~~~~~~~~
%%  GENERAL PACKAGES:
%%% ~~~~~~~~~~~~~~~~~~~~~~~~~~~~~~~~~

\usepackage{paralist}
%\usepackage{calc}
%\usepackage{titlesec}

%%% ~~~~~~~~~~~~~~~~
%%  BETTER SOURCE CODE LISTINGS:
%%  NOTE: the minted package is ALREADY LOADED by:
%%        09-4-org-export-conf.el Just USE it HERE...

%% Use Console Command:
%%    $ pygmentize -L styles
%% to see a list of all possible "minted" code listing styles.
%% Then change {style-name} below to try any of them out...
%%    tango   - Typical...
%%    perldoc - A Nice One...
%%    xcode   - Similar to perldoc (even nicer)

\setminted{style=xcode,frame=leftline}
\setminted[r]{linenos=true}

%%% ~~~~~~~~~~~~~~~~~~~~~~~~~~~~~~~~~
%%  LANDSCAPE & MARGINS:
%%% ~~~~~~~~~~~~~~~~~~~~~~~~~~~~~~~~~

%% This allows using \begin{landscape} & \end{landscape} 
%% Before and After WIDE Figures, Tables, images, etc...
\usepackage{pdflscape}

\usepackage[
	margin=1.5cm  % Equal margin on all sides
%	showframe     % Uncomment to show frames around the margins for debugging purposes
]{geometry}

%%% ~~~~~~~~~~~~~~~~~~~~~~~~~~~~~~~~~
%% FONTS:
%% NOTE: [T1]{fontenc} is included and LOADED
%%       by org-mode [DEFAULT-PACKAGES]
%%       (org-latex-default-packages-alist)

%%% ~~~~~~~~~~~~~~~~
%% REFERENCE BOOK FONT SPECS:

\usepackage{roboto}
\usepackage{palatino}

%\usepackage{libertine}
%\usepackage{libertinust1math}
%\usepackage{gfsneohellenic}
%\renewcommand{\sfdefault}{neohellenic}

%\titleformat*{\section}{\Large\sffamily}
%\titleformat*{\subsection}{\large\sffamily}
%\titleformat*{\subsubsection}{\normalsize\sffamily}

%%% ~~~~~~~~~~~~~~~~~~~~~~~~~~~~~~~~~
%%. TABLES
%%% ~~~~~~~~~~~~~~~~~~~~~~~~~~~~~~~~~

%% Book Like Table Formatting
\usepackage{booktabs}

%% Set table column padding to be proportional to text width
\setlength{\tabcolsep}{0.0075\textwidth}

%%% ~~~~~~~~~~~~~~~~~~~~~~~~~~~~~~~~~
%% DEFAULT PARAGRAPH SETTINGS %%
%% Yeah... I read the manual and their reasons for indenting paragraphs,
%% BUT this is a Technical Book... NOT a Fiction Novel...
%%% ~~~~~~~~~~~~~~~~~~~~~~~~~~~~~~~~~

\setlength{\headwidth}{\textwidth}
\addtolength{\headwidth}{.382\foremargin}

%%% ~~~~~~~~~~~~~~~~
%% BLOCK PARAGRAPHS No Indent:

\setlength{\parindent}{0pt} % Paragraph indentation
\setlength{\parskip}{6pt} % Vertical space between paragraphs

%%% ~~~~~~~~~~~~~~~~~~~~~~~~~~~~~~~~~
%% SPECIAL PACKAGES:
%%% ~~~~~~~~~~~~~~~~~~~~~~~~~~~~~~~~~

%\usepackage{soul}
%\usepackage{pdfpages}
%\usepackage[backend=bibtex,sorting=none]{biblatex}

%%% ~~~~~~~~~~~~~~~~
%%  GRAPHICS PACKAGES: 

\usepackage{svg}

%%% ~~~~~~~~~~~~~~~~
%%  MUSIC PACKAGES: 

\usepackage[minimal]{leadsheets}
\useleadsheetslibraries{musicsymbols}
\useleadsheetslibraries{chords}

%%%% ~~~~~~~~~~~~~~~~~~~~~~~~~~~~~~~~~~~~~~~
%%%% CUSTOM CODE FROM: MemoirChapStyles.pdf
%%%% ~~~~~~~~~~~~~~~~~~~~~~~~~~~~~~~~~~~~~~~



%%%% ~~~~~~~~~~~~~~~~~~~~~~~~~~~~~~~~~~~~~~~
%% FINAL PREAMBLE DIRECTIVES:
%%%% ~~~~~~~~~~~~~~~~~~~~~~~~~~~~~~~~~~~~~~~

\chapterstyle{verville}

%%%% END: ~/.emacs.d/Docs/TeX/ref-book-setup.tex
%%%% ~~~~~~~~~~~~~~~~~~~~~~~~~~~~~~~~~~~~~~~~~~~~~~~~~~~~~~~~~~~~~~~~~~~~~~~~~~~~
