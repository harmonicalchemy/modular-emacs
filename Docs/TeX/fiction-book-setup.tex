%%%% ~~~~~~~~~~~~~~~~~~~~~~~~~~~~~~~~~~~~~~~~~~~~~~~~~~~~~~~~~~~~~~~~~~~~~~~~~~~~
%%   File:        ~/.emacs.d/Docs/TeX/fiction-book-setup.tex
%%   Ref:         <https://github.com/harmonicalchemy/modular-emacs>
%%   Purpose:     LaTeX Setup Include File for Fiction Books
%%   Author:      Alisha Awen
%%   Maintainer:  Alisha Awen
%%   Created:     2023-002-18
%%   Updated:     2023-002-18

%%   This program is free software: you can redistribute it and/or modify
%%   it under the terms of the GNU General Public License as published by
%%   the Free Software Foundation, either version 3 of the License, or
%%   (at your option) any later version.

%%   This program is distributed in the hope that it will be useful,
%%   but WITHOUT ANY WARRANTY; without even the implied warranty of
%%   MERCHANTABILITY or FITNESS FOR A PARTICULAR PURPOSE.  See the
%%   GNU General Public License for more details.

%%   You should have received a copy of the GNU General Public License
%%   along with this program.  If not, see <https://www.gnu.org/licenses/>.

%%%  CUSTOM USAGE:

%%   This Configuration defines the: BLUE BOX CHAPTER HEADING STYLE
%%   To Customize: Copy/Clone ~/.emacs.d/Docs/PubOps/org-templates (which contains
%%   this file) to your .org doc's MASTER FOLDER. Then, modify your cloned copy of
%%   this file...
%%%% ~~~~~~~~~~~~~~~~~~~~~~~~~~~~~~~~~~~~~~~~~~~~~~~~~~~~~~~~~~~~~~~~~~~~~~~~~~~~

%%% ~~~~~~~~~~~~~~~~~~~~~~~~~~~~~~~~ PREAMBLE ~~~~~~~~~~~~~~~~~~~~~~~~~~~~~~~ %%%

\usepackage{parskip}
\usepackage{paralist}
\usepackage[hyperref,x11names]{xcolor}
\usepackage[colorlinks=true,urlcolor=SteelBlue4,linkcolor=Firebrick4]{hyperref}
\usepackage{color,calc}

%%% ~~~~~~~~~~~~ FONT SPECS ~~~~~~~~~~~~~

\usepackage[LGR, T1]{fontenc}
\usepackage{libertine}
\usepackage{libertinust1math}
\usepackage{gfsneohellenic}

\renewcommand{\sfdefault}{neohellenic}

\usepackage{titlesec}

\titleformat*{\section}{\Large\sffamily}
\titleformat*{\subsection}{\large\sffamily}
\titleformat*{\subsubsection}{\normalsize\sffamily}

%%% ~~~~~~~~ CHAPTER NUMBER BOX ~~~~~~~~~

\newsavebox{\ChpNumBox}
\definecolor{ChapBlue}{rgb}{0.00,0.65,0.65}

\makeatletter

\newcommand*{\thickhrulefill}{%
   \leavevmode\leaders\hrule height 1\p@ \hfill \kern \z@
}

\newcommand*\BuildChpNum[2]{%
   \begin{tabular}[t]{@{}c@{}}
   \makebox[0pt][c]{#1\strut} \\[.5ex]
   \colorbox{ChapBlue}{%
      \rule[-10em]{0pt}{0pt}%
      \rule{1ex}{0pt}\color{black}#2\strut
      \rule{1ex}{0pt}}%
   \end{tabular}
}

%%% ~~~~~~~~~~~~~~~~~~ Make BLUE BOX CHAPTER HEADING STYLE ~~~~~~~~~~~~~~~~~~ %%%

\makechapterstyle{BlueBox}{%
   \renewcommand{\chapnamefont}{\large\scshape}
   \renewcommand{\chapnumfont}{\Huge\bfseries}
   \renewcommand{\chaptitlefont}{\raggedright\Huge\bfseries}
   \setlength{\beforechapskip}{20pt}
   \setlength{\midchapskip}{26pt}
   \setlength{\afterchapskip}{40pt}
   \renewcommand{\printchaptername}{}
   \renewcommand{\chapternamenum}{}

   \renewcommand{\printchapternum}{%
      \sbox{\ChpNumBox}{%
         \BuildChpNum{\chapnamefont\@chapapp}%
         {\chapnumfont\thechapter}
      }
   }

   \renewcommand{\printchapternonum}{%
      \sbox{\ChpNumBox}{%
         \BuildChpNum{\chapnamefont\vphantom{\@chapapp}}%
         {\chapnumfont\hphantom{\thechapter}}
      }
   }

   \renewcommand{\afterchapternum}{}

   \renewcommand{\printchaptertitle}[1]{%
      \usebox{\ChpNumBox}\hfill

      \parbox[t]{\hsize-\wd\ChpNumBox-1em}{%
         \vspace{\midchapskip}%
         \thickhrulefill\par
         \chaptitlefont ##1\par
      }
   }%
} %% END BlueBox Chapter Style...

\makeatother
%%% ~~~~~~~~~~~~~~~~~~~~~~~~~~~~~~~~~~~~~

%%% ~~~~~~~~~~~~~~~~~~~~~~~~~~~~ BEGIN DOCUMENT ~~~~~~~~~~~~~~~~~~~~~~~~~~~~~ %%%

\AtBeginDocument{
\nonzeroparskip
\frontmatter
}

\chapterstyle{BlueBox}

%%%% END: ~/.emacs.d/Docs/TeX/fiction-book-setup.tex
%%%% ~~~~~~~~~~~~~~~~~~~~~~~~~~~~~~~~~~~~~~~~~~~~~~~~~~~~~~~~~~~~~~~~~~~~~~~~~~~~
